
 \documentclass{article}
\usepackage[utf8]{inputenc}
\usepackage{amssymb}
\usepackage{dcolumn}
\usepackage{amsfonts}
\usepackage{amsmath}
\usepackage[a4paper, total={6in, 8in}]{geometry}

\usepackage{amsmath,amssymb,array,amsthm}
\usepackage{amsfonts}

\usepackage{parskip}

\usepackage{tikz-cd}
\usepackage{graphicx}

\newcommand*{\field}[1]{\mathbb{#1}}%

\usepackage{comment}
\usepackage{listings}
\usepackage[mathletters]{ucs}
\usepackage[utf8]{inputenc}


\makeatletter
\newcommand\xleftrightarrow[2][]{%
  \ext@arrow 9999{\longleftrightarrowfill@}{#1}{#2}}
\newcommand\longleftrightarrowfill@{%
  \arrowfill@\leftarrow\reloverline\rightarrow}
\makeatother

\newtheorem*{remark}{Remark}
\newtheorem*{theorem}{Theorem}
\newtheorem*{corollary}{Corollary}
\newtheorem*{lemma}{Lemma}
\newtheorem*{definition}{Definition}
\newtheorem*{proposition}{Proposition}
\newtheorem*{claim}{Claim}

\usepackage{quiver}

\newcommand{\normal}{\textnormal}
\newcommand{\lam}{\mathbb{\lambda}}
\newcommand{\R}{\mathbb{R}}
\newcommand{\RP}{\mathbb{R}\text{P}}
\newcommand{\Q}{\mathbb{Q}}
\newcommand{\Z}{\mathbb{Z}}
\newcommand{\N}{\mathbb{N}}
\newcommand{\C}{\mathbb{C}}
\newcommand{\ps}{\psi}
\newcommand{\then}{\Rightarrow}
\newcommand{\liff}{\Leftrightarrow}
\newcommand{\lang}{\langle}
\newcommand{\rang}{\rangle}
\newcommand{\proves}{\vdash}
\newcommand{\goth}{\mathfrak}
\newcommand{\Id}{\text{Id}}
\newcommand{\refl}{\text{refl}}
\newcommand{\proj}{\text{proj}}
\newcommand{\colim}{\text{colim}}
\newcommand{\fib}{\twoheadrightarrow}
\newcommand{\cofib}{\rightarrowtail}
\newcommand{\T}{\mathtt{T}}
\newcommand{\CxtCat}{\mathtt{CtxCat}}
\newcommand{\CwA}{\mathtt{CwA}}
\newcommand{\CompCat}{\mathtt{CompCat}}
\newcommand{\SplCompCat}{\mathtt{SplCompCat}}
\newcommand{\Trb}{\mathtt{Trb}}
\newcommand{\STLC}{\mathtt{STLC}}
\newcommand{\CCC}{\mathtt{CCC}}
\newcommand{\Geb}{\mathtt{Geb}}

\newcommand{\into}{\hookrightarrow}
\title{The Geb Category}
\author{Artem Gureev}
\date{}




\begin{document}

\maketitle

The document provides the outline of the intended semantics of the Geb programming language without dependent types, i.e. of the initial model of Geb.

We have two ways of describing the Geb category, one via the universal property and one via explicit construction.

Recall that category theory is an essentially algebraic theory and, moreover, the theory of finite products and coproducts in a category is essentially algebraic. Finally, adding the axiom of distributivity is also an equationally specified operation. Hence we have the universal characterization:

\begin{definition}
The $\Geb$ category is the initial model of an essentially algebraic theory of categories with finite products, coproducts, and distributivity. In other words, it is a free model of said theory on the empty set.
\end{definition}

This provides the characterization via the universal property of the free$\dashv$forgetful adjunction.

We can also specify the category explicitly.

\begin{definition}
Define $G0_{n}$ for $n \in \N$ inductively:

$G0_0 := \{I, T\}$

$G0_{n+1} := G0_n \cup \{a + b \; \vert \; a, b \in G0_n \} \cup \{a \times b \; \vert \; a, b \in G0_n\}$ 

Define $G_{obj}$ as a colimit of  
% https://q.uiver.app/?q=WzAsNixbMCwwLCJHXzAiXSxbMSwwLCJHXzEiXSxbMywwLCJHX24iXSxbNCwwLCJHX3tuKzF9Il0sWzIsMCwiLi4uIl0sWzUsMCwiLi4uIl0sWzIsMywiIiwwLHsic3R5bGUiOnsidGFpbCI6eyJuYW1lIjoiaG9vayIsInNpZGUiOiJ0b3AifX19XSxbMSw0LCIiLDAseyJzdHlsZSI6eyJ0YWlsIjp7Im5hbWUiOiJob29rIiwic2lkZSI6InRvcCJ9fX1dLFs0LDIsIiIsMCx7InN0eWxlIjp7InRhaWwiOnsibmFtZSI6Imhvb2siLCJzaWRlIjoidG9wIn19fV0sWzAsMSwiIiwwLHsic3R5bGUiOnsidGFpbCI6eyJuYW1lIjoiaG9vayIsInNpZGUiOiJ0b3AifX19XSxbMyw1LCIiLDAseyJzdHlsZSI6eyJ0YWlsIjp7Im5hbWUiOiJob29rIiwic2lkZSI6InRvcCJ9fX1dXQ==
\[\begin{tikzcd}
	{G0_0} & {G0_1} & {...} & {G0_n} & {G0_{n+1}} & {...}
	\arrow[hook, from=1-4, to=1-5]
	\arrow[hook, from=1-2, to=1-3]
	\arrow[hook, from=1-3, to=1-4]
	\arrow[hook, from=1-1, to=1-2]
	\arrow[hook, from=1-5, to=1-6]
\end{tikzcd}\]
or equivalently $\bigcup_{n \in \N} G0_n$, or the initial algebra in $Set$ of the endofunctor $2X^2+2$
\end{definition}

\begin{definition}
Define $G1_n$ and $(dom, cdom) : G1_n \times G1_n \to G_{obj}$ for $n \in \N$ by induction on $n$.
\begin{align*}
G1_{0} := &\{1_x \;\vert\; x \in G_{obj}\} \\
&\cup \{!_{I, x} \;\vert\; x \in G_{obj}\}  \\
&\cup \{!_{x, T} \;\vert\; x \in G_{obj}\} \\
&\cup  \{\pi_{k, x, y} \;\vert\; k \in 2,  x, y \in G_{obj}\} \\
&\cup \{i_{k, x, y} \;\vert\; k \in 2, x, y \in G_{obj}\}\\  
&\cup  \{d_{x, y, z} \;\vert\; x, y, z \in G_{obj}\}
\end{align*} 
with the evident (co)domain functions where $d$ is supposed to be the distributivity map, i.e. $dom(d_{x, y, z)}) = x \times (y + z)$ and $codom(d_{x, y, z}) = (x \times y) + (x \times z)$
\begin{align*}
G1_{n+1} := G1_n &\cup \{f \circ g \;\vert\; f, g \in G1_{n}, codom(g)=dom(f)\} \\
&\cup \{\lang f, g \rang \;\vert\; f, g \in G1_{n}\}\\
&\cup \{[f, g] \; \vert \; f, g \in G1_{n} \}
\end{align*}

similarly with evident (co)domain functions.

Define $G_{mor} := \bigcup_{n \in \N} G1_{n}/\sim$ where $\sim$ is the smallest equivalence relation spanned by $(f \circ g) \circ k \sim f \circ (g \circ k)$, $\pi_{1, x, y} \circ \lang f, g \rang \sim f$, $\pi_{2, x, y} \circ \lang f, g \rang \sim g$, $[f, g] \circ i_{1, x, y} \sim f$, $[f, g] \circ i_{2, x, y} \sim g$, $1_{x} \circ f \sim f$, $f \circ 1_{x} \sim f$, and making $d_{x, y, z}$ the inverse of the evident morphism.
\end{definition}

\begin{definition}
Let $\Geb$ designate the category whose class of objects is $G_{obj}$ and set of morphisms $G_{mor}$ with the evident $(co)dom$ functions defined inductively, composition given by $comp ([f] , [g]) := [f \circ g]$, and $\{[1_{x}] \; \vert\; x \in G_{obj}\}$ being the set of identities.
\end{definition}


\end{document}